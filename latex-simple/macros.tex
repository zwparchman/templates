%%%%%%%%%%%%%%%%%%%%%%%%%%%%%%%%%%%%%%%%%
% LaTeX Macros
%%%%%%%%%%%%%%%%%%%%%%%%%%%%%%%%%%%%%%%%%


%
% Setup a few helpful formatting macros
%


% Use urlstyle to protect for usage with functions containing underscores
\newcommand\func{\begingroup \urlstyle{tt}\Url}
\newcommand\variable{\begingroup \urlstyle{tt}\Url}
\newcommand\envvar{\begingroup \urlstyle{tt}\Url}
\newcommand\file{\begingroup \urlstyle{tt}\Url}

\newcommand{\namespace}[1]{\texttt{#1}}

%\newcommand{\variable}[1]{\texttt{#1}}
% \newcommand{\file}[1]{\texttt{#1}}
\newcommand{\cmd}[1]{\texttt{#1}}
\newcommand{\function}[1]{\texttt{#1}}
\newcommand{\manpage}[2]{\texttt{#1(#2)}}
\newcommand{\user}[1]{\texttt{#1}}
\newcommand{\hostname}[1]{\texttt{#1}}


 % TJN: Mirror the acronym macros with the Glossary macros
 %      so it is earier to use in the manner I tend to use the package.
 % This forces the "full" verison of glossary, i.e., first use string,
 % Example: \glsfull{ft} => "fault tolerance~(FT)".
\newcommand{\glsfull}[1]{\glsreset{#1}\gls{#1}}

%-----
% Changebar macro
%  Usage:
%     \changebegin{Thomas} ...my modified lines...  \changeend{}

\newcommand{\changebegin}[1]{\marginpar[\hspace*{-60pt}\mbox{\hspace*{10pt} 
 $\top$ \tiny ({#1})}]{\mbox{$\top$ \tiny ({#1})}}}
\newcommand{\changeend}[1]{\marginpar[\hspace*{-60pt}\mbox{\hspace*{10pt} 
 $\bot$ \tiny ({#1})}]{\mbox{$\bot$ \tiny ({#1})}}}
%
% To get rid of these change marks (without actually removing from text)
%\newcommand{\changebegin}[1]{} 
%\newcommand{\changeend}[1]{} 
%-----



%-----
%
% Discussion macro
%  Usage:
%     \beging{discuss} ...discussion/comments...  \end{discuss}

\newenvironment{discuss}{\begin{small}\begin{list}{}{}\item[]{\it \underline{\textcolor{cyan}{Discussion item:}}}
\addcontentsline{toc}{subsection}{\textcolor{cyan}{Discussion Item}}}{{\rm ({\it \textcolor{cyan}{End of discussion item.}})} \end{list}\end{small}}
%
%-----


%-----
%
% Review Text macro -- place this around text that needs 
%                      to be reviewed/revised. 
%                      (Note: Not sure how macro will work across sections.)
% Usage:
%   \begin{reviewme}
%     ...some text in the doc to review...
%   \end{reviewme}
%

\newenvironment{reviewme}{\begin{paragraph}{\textcolor{magenta}{\it Review text:}}
\addcontentsline{toc}{subsection}{\textcolor{magenta}{Review text:}}
\changebegin{\textcolor{magenta}{ReviewME}}}
{{\rm ({\textcolor{magenta}{\it End of review text.}})\changeend{\textcolor{magenta}{ReviewME}}}\end{paragraph}}
%
% Alt. version: much more compact (doesn't show Begin/End text marks).
%
%\newenvironment{reviewme}{\addcontentsline{toc}{subsection}{\textcolor{magenta}{Review text:}}
%\changebegin{\textcolor{magenta}{ReviewME}}}{\changeend{\textcolor{magenta}{ReviewME}}}
%
%-----


%%%%%%%%%%%%%%%%%%%%%%
% Create compact lists
%%%%%%%%%%%%%%%%%%%%%%
% Usage: 
%
%    [DEFAULT]                       [ALT-VERSION]
%   \begin{itemize*}               \begin{compactitemize}
%      \item xxx       -OR-           \item xxx
%   \end{itemize*}                 \end{compactitemize}
%
% Add alternate version with more descriptive LaTeX keywords,
% the package provides the "...*" (star) version by default.
\makecompactlist{compactenumerate}{enumerate}
\makecompactlist{compactitemize}{itemize}
\makecompactlist{compactdescription}{description}


%-----
%
% Attention/Notice macro
%  Usage:
%     \beging{AttentionNote} ...note/comments...  \end{AttentionNote}
%
\newenvironment{AttentionNote}{\begin{quote}\begin{list}{}{}\item[] 
    \hfil\rule{0.8\textwidth}{.4pt}\hfil \\
    {\bf Notice:\\}}{{
    \\ \hfil\rule{0.8\textwidth}{.4pt}\hfil \\
    } \end{list}\end{quote}}
%
%-----

%-----
%
% ToDo macro
%   Usage:
%      \xxxtodo{INITIALS}{Todo text here}
%      \xxxtodoMargin{INITIALS}{Todo text for margin here}
%
%   Example
%      \xxxtodo{TJN}{This would be a good example for todo.}
%      \xxxtodoMargin{TJN}{And something in the margin}
%
%   Descr:
%      Places a numbered 'todo' with the given initials and text into
%      the document with a MS-Word style yellow comment box.
%      The "Margin" version, simply puts the todo box in the margin,
%      which can be helpful for general comments to a section/paragraph.
%      
%%
\newcounter{todocounter}
\newcommand{\todonum}[2][]
{\stepcounter{todocounter}\todo[#1]{\thetodocounter: #2}}
%
% ToDo comments
\newcommand{\xxxtodo}[2]{\todonum[color=yellow!30,inline,size=\footnotesize,caption=Comment~\thetodocounter]{#1: #2}}
\newcommand{\xxxtodoMargin}[2]{\todonum[color=yellow!30,size=\scriptsize,noline,caption=MarginComment~\thetodocounter]{#1: #2}}
%
%-----


